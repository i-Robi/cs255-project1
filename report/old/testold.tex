\documentclass[10pt,twocolumn]{article}
\usepackage{amsmath}
\usepackage{stmaryrd}
\usepackage{parskip}
\setlength{\parindent}{0pt}
\title{
  CS 255 -- Project 1 (Part 1) }
\author{
  Sammy El Ghazzal\\
  (SUNetID: selghazz)
  \and
  S\'ebastien Robaszkiewicz\\
  (SUNetID: robinio)
}
\date{}
\begin{document}
  \maketitle


%%------------------------------------------------------------------
%%------------------------------------------------------------------
%%
%% Implementation
%%
%%------------------------------------------------------------------
%%------------------------------------------------------------------


\section{Implementation}

%%------------------------------------------------------------------
%% Maintaining a secure database
%%------------------------------------------------------------------

\subsection{Maintaining a secure database}

In the rest of the document, the \textit{database} will refer to a JSON object called \texttt{keys}. In this JSON object, the key-values pairs are as follows:
\begin{itemize}
  \item Key: it is the name of a Facebook group a user belongs to;
  \item Value: it is the (cryptographic) key used to encrypt the messages in this group.
\end{itemize}

\subsubsection{Initialization}

When a user connects to Facebook \textit{for the first time}, we use the following process.

\paragraph{Password}
The first time a user connects to Facebook, we ask to set a password to encrypt the database thanks to the function \texttt{prompt()}.

\paragraph{Salt}\label{salt}
Once the password has been set, we generate the salt (a \texttt{bitArray} of length~128) with the function \texttt{GetRandomValues}. We convert the salt to the \texttt{base64} string format, and store this string in plaintext in the \texttt{localStorage}.

\paragraph{Key}
We generate the key (another \texttt{bitArray}) thanks to the function \texttt{sjcl.misc.pbkdf2}: the key is derived from the password and the salt mentioned above. We convert the key to the \texttt{base64} string format, and we store it in \texttt{sessionStorage}.

\paragraph{Encrypting the database}
We chose to encrypt the entire \texttt{keys} JSON object by converting it to a string, and encrypting the whole string with the key mentioned above. We use the \textit{Randomized Counter Mode} construction on top of AES, as explained in section~\ref{encryption-decryption}. We store this encrypted JSON in \texttt{localStorage}.


\subsubsection{Regular use}

When a user comes \textit{back} to Facebook (\emph{i.e.} when the plaintext salt and the encrypted database already exist in \texttt{localStorage}), we ask for the password in order to decrypt the database. 

\paragraph{Is the password correct?}

When the user enters his password, we generate a key just like previously: we use the function \texttt{sjcl.misc.pbkdf2} with this password, and the salt which is already stored in \texttt{localStorage}. In order to be able to check that the password
is correct, we added a dummy entry in the database, \texttt{00000000 => 0000}.
Then, we try to decrypt the database with the generated key: if we can read the entry \texttt{00000000} and its associated value \texttt{0000}, then the entered password is the right password. If not, we ask for the password again. If the user doesn't want to enter a password, we terminate the script.


%%------------------------------------------------------------------
%% Generating new keys
%%------------------------------------------------------------------

\subsection{Generating new keys}
The function \texttt{GenerateKeys} is pretty simple as it relies heavily on the \texttt{GetRandomValues} function. We chose to generate 256-bit keys (even though as we saw in class, the 256-bit AES implementation may be attacked in $2^{112}$ (toto verif this number) under certain conditions). For convenience, we store a \texttt{base64} encoding of the key in the database \texttt{keys}.


%%------------------------------------------------------------------
%% Encryption and decryption functions that provide CPA security for Facebook group messages
%%------------------------------------------------------------------

\subsection{Encryption and decryption functions that provide CPA security for Facebook group messages}\label{encryption-decryption}

\paragraph{Choice of the block cipher}
We chose to implement the \emph{Randomized Counter Mode} construction on top of AES because of the advantages it presents compared to other constructions (CBC in particular):
\begin{itemize}
\item It is parallelizable (although in general Facebook messages will not be long enough to justify the need for a parallel system)
\item In case a block is corrupted, only one block of ciphertext will be corrupted\footnote{In our particular case, the fact that one block is corrupted is sufficient to throw an error in the JS script and therefore stop the decryption / encryption taking place, but this remark goes beyond the scope of this assignment.}
\item Its security bound is better (see Section~\ref{sec:security} for more details)
\end{itemize}
For the IV, we use a 128-bit nonce (which corresponds to exactly one block) chosen at random thanks to the function \texttt{GetRandomValues}. This nonce is concatenated at the beginning of the ciphertext.

\paragraph{Padding}
Although padding is not necessary when using \emph{Randomized Counter Mode}, our encryption and 
decryption schemes use padding whenever the encrypted message length is not a multiple of the block size and a dummy block otherwise. More precisely:
\begin{itemize}
\item If the plaintext has a length that is a multiple of the block size (namely 128 bits \emph{i.e.} 16 characters), we add a dummy block of 16 characters \texttt{f} at the end of the plaintext
For instance ``Hello nice world'' would be transformed into ``Hello nice worldffffffffffffffff'' before being encrypted.
\item If the plaintext has a length that has $1 \leq r \leq 15$ characters after the end of the last full block, then we add $16-r$ times the encoding of $15-r$ in hexadecimal\footnote{The fact that we chose to encode the “residual” length in hexadecimal is that it allows us to have only one character for double digit lengths.} at the end of the plaintext before the encryption.
For instance, the sentence ``Hello world'' which has 11 characters, would be padded as follows: ``Hello world4444''
\end{itemize}
When decrypting, we consider the string as if it had a length that is a multiple of the block length, and then we remove the number of characters indicated by the last character.

\paragraph{Encodings}
We experienced some trouble with the encodings of strings. As a result, the encoding of the plaintext and ciphertext are different, namely the plaintext is considered to be a \texttt{utf8String} and the ciphertext is encoded in \texttt{base64} string format. When decrypting, instead of splitting the ciphertext into 16-character chunks, we split it into 24-character chunks.



%%------------------------------------------------------------------
%%------------------------------------------------------------------
%%
%% Security
%%
%%------------------------------------------------------------------
%%------------------------------------------------------------------

\section{Security}
\label{sec:security}
%%------------------------------------------------------------------
%% Security of the key storage
%%------------------------------------------------------------------

\subsection{Security of the key storage}

Each step of the process is secure:
\begin{itemize}
  \item The salt is generated at random based on Chrome's \texttt{window.crypto.getRandomValues} function, which provides cryptographically secure random numbers
  \item Key generated from that salt along with the password using PBKDF2, as recommended by RSA's PKCS \#5 standard
  \item We encrypt / decrypt the database using the same algorithms as for encrypting / decrypting message, please refer to section~\ref{security-encryption-decryption} for security details about those algorithms
\end{itemize}

That way, the only information that an attacker has about the database is its total size.

In summary, the salt and the securely encrypted database are stored in \texttt{localStorage} (which is the only thing the attacker can access), whereas the database key (which is generated securely) is stored in \texttt{sessionStorage} and the attacker will never see it. Note that during the whole process, we never store the password anywhere.



%%------------------------------------------------------------------
%% Security of the key generation
%%------------------------------------------------------------------

\subsection{Security of the key generation}

The key generation step relies on the \texttt{GetRandomValues} function. The design of this function ensures that information encoded by such a key is indistinguishable from random (unlike a function from the javascript \texttt{Math} library for instance).




%%------------------------------------------------------------------
%% Security of the encryption / decryption steps
%%------------------------------------------------------------------

\subsection{Security of the encryption / decryption steps}\label{security-encryption-decryption}

There are toto main steps to ``prove'' the security\footnote{In the remainder of this write-up, we use ``is secure''
instead of ``supposed to be secure''.}  of our encryption and decryption schemes:
\begin{itemize}
\item The underlying primitive we used for block cipher is \emph{AES-256} which is supposed to be a secure PRP
\item The construction we used on top of \emph{AES-256} is \emph{Randomized Counter Mode}, which is secure under CPA as long as the nonce space is large enough and the nonce is chosen at random (we took a 128-bit random nonce so it is the case here).
More precisely, if $q$ is the number of queries the adversary $A$ is allowed to do, we know that there exists a PRF adversary $B$, such that:
\begin{equation}\label{eq:sec1}
 \text{Adv}_{CPA}[A,E_{CTR}] \leq 2 \text{Adv}_{PRF}[B,\text{AES}] +  \frac{2q^2L}{|X|}
\end{equation}
where in our case $|X| = 2^{128}$ and therefore, if $A$ is able to break $E_{CTR}$ under the assumption that $\frac{2q^2L}{|X|}$ is negligible\footnote{This assumption basically means that the user of the script should change his key when the term $\frac{2q^2L}{|X|}$ is non-negligible, \emph{i.e.} that the user uses the library correctly.}, then it would mean that $\text{Adv}_{PRF}[B,\text{AES}]$ is non-negligible, \emph{i.e.} that adversary $B$ would be able to break AES, which we suppose to be impossible.
\paragraph{Important note:} To argue for the security of our encryption and decryption schemes, we make the important assumption that $\frac{2q^2L}{|X|}$ is negligible, \emph{i.e.} that the user changes the key whenever $q^2L$ is ``too big''. In our script, we \emph{do not check} that this is indeed the case\footnote{Justin Chen told me that it was ok not to make this checking.}. Just to have an idea of how many messages can be sent using one key, let us do a quick calculation: Facebook requires the message length to be less than 10000 characters long, that is $L$ = 625 blocks. Then we compute that the number of such messages sent $q$ should be such that (we say that the advantage is negligible if it is less than $\epsilon = 2^{-80}$)
\begin{equation*}
q \leq \sqrt{\frac{\epsilon|X|}{2L}}  = q_{max} \Leftrightarrow q_{max} \approx 470000
\end{equation*}
That is, for the encryption / decryption scheme to become insecure, the user would need to send around 470000 messages of 10000 characters each. Saying that this is too much for a single group is probably reasonable, even if to be perfectly secure, it would be better to implement a system that keeps track of the number of messages being sent through the group.
\end{itemize}



% \section{Other things}
% \paragraph{Using a web browser to do cryptography}
% 
% \paragraph{Potential attacks}
% There are a couple of potential attacks on our script:
% \begin{itemize}
%  \item Brute-force attack on the password: having at his disposal the salt and the encrypted in the \texttt{localStorage}, a pirate could try to decrypt the database by trying a large number
%  of passwords and see if the key generated from the salt and the password manage to decrypt the database. This is a serious attack as it can in theory be realized in parallel (by copying the files
%  to several machines) and therefore if the password is not too strong, the database can be decrypted in a relatively small amount of time.
% \end{itemize}


\end{document}


